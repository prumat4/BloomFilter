\documentclass[12pt]{article}
\usepackage[english,ukrainian]{babel}
\usepackage[letterpaper,top=2cm,bottom=2cm,left=3cm,right=3cm,marginparwidth=1.75cm]{geometry}
\usepackage{amsmath}
\usepackage{graphicx}
\usepackage{booktabs}
\usepackage[colorlinks=true, allcolors=blue]{hyperref}
\usepackage{pgfplots}
\pgfplotsset{compat=1.17}

\begin{document}

\section{Вступ}
\quad Фільтр Блума - структура даних, яка реалізовує пошук улументів у множинах довільної природи. Вона використовує бітовий масив та набір незалежних геш-функцій для реалізації структури.

\section{Використання \texttt{std::bitset} у BloomFilter}
\quad Фільтр Блума використовує \texttt{std::bitset} для позначення наявності елементів в множині. В порівнянні з масивами булів, \texttt{std::bitset} є переважнішим в багатьох аспектах.

\quad Перш за все, використання \texttt{std::bitset} дозволяє точніше та ефективніше використовувати пам'ять, оскільки кожен біт відповідає стану одного елемента. Це дає перевагу у використанні пам'яті порівняно з масивами булів, які можуть займати більше місця через внутрішнє вирівнювання та додаткові біти інформації у кожному булі.

\quad До того ж, \texttt{std::bitset} має фіксований розмір, що робить його простішим у використанні та реалізації порівняно з динамічно змінюваними масивами булів.

\quad У випадку порівняння з масивами цілих чисел, використання \texttt{std::bitset} зазвичай є більш простим та зрозумілим варіантом, пропонуючи аналогічну ефективність використання пам'яті.

\newpage
\section{Експеримент}

\begin{table}[htbp]
\centering
\begin{tabular}{ccc}
\toprule
$s$ & \textbf{Фактор} & \textbf{Помилка} \\
\midrule
13 & 0.05 & 0.00322193824 \\
6 & 0.1 & 0.1422562279 \\
4 & 0.15 & 0.2887196545 \\
3 & 0.2 & 0.3451322751 \\
2 & 0.25 & 0.2830952381 \\
2 & 0.3 & 0.4355555556 \\
1 & 0.35 & 0.6875 \\
1 & 0.4 & 0.5408333333 \\
1 & 0.45 & 0.725 \\
1 & 0.5 & 0.5674242424 \\
\bottomrule
\end{tabular}
\caption{Результати експерименту}
\label{tab:experiment}
\end{table}

\textbf{Середня ймовірність помилки}: 0.3018738465

\begin{figure}[htbp]
    \centering
    \begin{tikzpicture}
        \begin{axis}[
            xlabel=$s$,
            ylabel={Фактор},
            zlabel={Помилка},
            width=0.8\textwidth,
            height=8cm,
            grid=major,
            view={120}{40},
        ]

        \addplot3[surf] coordinates {
            (13, 0.05, 0.00322193824)
            (6, 0.1, 0.1422562279)
            (4, 0.15, 0.2887196545)
            (3, 0.2, 0.3451322751)
            (2, 0.25, 0.2830952381)
            (2, 0.3, 0.4355555556)
            (1, 0.35, 0.6875)
            (1, 0.4, 0.5408333333)
            (1, 0.45, 0.725)
            (1, 0.5, 0.5674242424)
        };
        \end{axis}
    \end{tikzpicture}
    \caption{Залежність помилки від $s$ та фактора}
    \label{fig:error_plot}
\end{figure}

\section{Висновок}
\quad Експеримент показав, що зменшення параметра $s$ при фіксованому факторі веде до збільшення помилки. Також, із збільшенням значення фактору помилка також зростає.
\bibliographystyle{alpha}

\end{document}
